\documentclass[czech,semestral]{diploma}
% Dalsi doplnujici baliky maker
\usepackage[autostyle=true,czech=quotes]{csquotes} % korektni sazba uvozovek, podpora pro balik biblatex
\usepackage[backend=biber, style=iso-numeric, alldates=iso]{biblatex} % bibliografie
\usepackage{dcolumn} % sloupce tabulky s ciselnymi hodnotami
\usepackage{subfig} % makra pro "podobrazky" a "podtabulky"
\usepackage[cpp]{diplomalst}

% Zadame pozadovane vstupy pro generovani titulnich stran.
\ThesisAuthor{Bc. Lukáš Moravec}

\ThesisSupervisor{Ing. Radoslav Fasuga, Ph.D.}

\CzechThesisTitle{Systém pro tvorbu podkladu pro on-line reklamní kampaně}

% \EnglishThesisTitle{Tools for Automated Generation of Promo Graphic Sources and Videos}

\SubmissionYear{2022}


% \Acknowledgement{Rád bych na tomto místě poděkoval vedoucímu bakalářské práce, Ing.~Radoslavovi~Fasugovi,~Ph.D., za věnovaný čas, vstřícnost a ochotu ke konzultacím při vytváření této práce. Dále bych chtěl také poděkovat své přítelkyni a rodině za neustálou podporu.}

% \CzechAbstract{Tato bakalářská práce je cílená na problematiku automatizovaného generování reklamních bannerů. V první části se zaměřuje na to, co to vlastně reklama je, jak funguje a nejčastější způsoby toho, jak využít různé formy reklamy k dosažení svých cílů. Dále popisuje tvorbu bannerů a podmínky pro nasazení do různých reklamních sítí. Práce analyzuje a porovnává existující nástroje k tvorbě reklamních materiálů, jejich možnosti automatizace tvorby, následně pojednává o tom, jak vytvořit nástroj, který by automatizaci tvorby bannerů umožňoval a popisuje důležité funkce úprav nezbytné k efektivnímu vytváření reklamních materiálů. Výsledná aplikace je jednostránková webová aplikace, sestavená za pomoci aplikačního rámce Angular a knihovny Konva.js.}

% \CzechKeywords{Reklama; Banner; Online grafika; Angular; Konva.js; SPA}

% \EnglishAbstract{This bachelor's thesis is aimed at the problematics of automated generation of advertising banners. In first part, the thesis focuses on what an advertisement really is, how it works and the most common ways of how to use different forms of advertising to achieve it's goals. It also describes the creation of banners and conditions for deployment in advertising networks. It analyses and compares existing tools to create advertising materials and their options of automated creation. It then discusses on how to create a tool, which would enable automated creation of banners. The thesis describes important editing features necessary to effective creation of advertising materials. The final application is a single page web application build with Angular framework and Konva.js library}

% \EnglishKeywords{Advertisement; Banner; Online graphics; Angular; Konva.js; SPA}

\addbibresource{bibliography.bib}

% Novy druh tabulkoveho sloupce, ve kterem jsou cisla zarovnana podle desetinne carky
\newcolumntype{d}[1]{D{,}{,}{#1}}

% Zacatek dokumentu
\begin{document}

% Nechame vysazet titulni strany.
\MakeTitlePages

\chapter{Úvod}
Možnosti internetové inzerce převyšujicí tradiční formy reklamy. Vystavit svou firmu pozornosti až tisíců potenciálních zákazníků během pár hodin je vysoké lákadlo pro investující firmy. Neustále zvyšující se nároky inzerentů stojí za vznikem komplexních systémů
pro správu online reklamních kampaní, které umožnují mít své nastavení pod naprostou kontrolou svých uživatelů. Aby dosah reklamy byl co nejširší, vyplatí se inzerovat na více platformách najednou. Těmi největšími platformami se staly
\emph{Google Ads, Facebook} nebo český \emph{Sklik}. Všechny tyto systémy poskytují jak své vlastní řešení pro správu kampaní, tak i aplikační rozhraní pro možnosti automatizace častých úkonů.
Aplikačním rozhraním se ale také otevírá brána pro vykonávání komplexních úkolů včetně kompletní správy kampaní napříč systémy. Tedy z jednoho místa, ovládat inzerci na několika platformách najednou.
Tato semestrální práce se zabývá vytvořením takovéhoto nástroje v online prostředí, který by umožňoval jednoduchou správu reklamních kampaní na více inzerčních systémů.


\chapter{Reklamní kampaně}
Reklamní kampaň lze definovat jako soubor komunikačních a reklamních aktivit, které mají za úkol oslovit veřejnost a splnit zvolený cíl. K uskutečnění kampaně se dá využít různých komunikačních
médií (internet, rozhlas, televize, \ldots). Tato kapitola popisuje základ tvorby reklamních kampaní a do hloubky rozebírá online kampaně a jejich možnosti.


\section{Základ pro tvorbu kampaní}
Pokud má být reklamní kampaň úspěšná, měla by být promyšlená. Tento úkol dokáže usnadnit úvaha nad následujícími body.

\subsection{Cíl kampaně}
Zvolení cíle zavisí na tom, co chceme kampaní dosáhnout. Typicky se jedná o zvýšení prodejnosti zboží nebo služeb, to ale nejsou jediné možnosti. Další mohou být například budování
povědomí o své značce (brand) nebo předání nějaké informace. Znát cíl pomáhá s další tvorbou kampaně a udává její směr.

\subsection{Rozpočet}
Finanční možnosti jsou jedním ze základních kritérií, ne-li nejdůležitějším. Od rozpočtu se odvíjí nejen celý zbytek plánování reklamní kampaně, ale i možná
doba trvání kampaně. Je velice pravděpodobné, že malý rozpočet se rychleji vyčerpe a proto je nutné kampaň vhodně rozložit.

\subsection{Publikum}
Cílové publikum je další důležitý faktor. Tato volba může znamenat rozdíl mezi dobrou nebo špatnou investicí. Různá sociodemografická kritéria pomohou lépe porozumět
na koho se kampaň směřuje. Toho lze využít při vytváření plánů komunikace, reklamy, nebo volby média.

\subsection{Média}
Různá média, různě působí na zvolené cílové publikum. I proto je dobře specifikované cílové publikum důležité. Pokud je cilovým publikem například starší populace, investice
do PPC reklamy může být značně nevýhodná. Dalším kritériem pro výběr komunikačního kanálů mohou být možnosti geologického cílení, zpětné vazby a statisky úspěšnosti.
S výběrem správného média může pomoct tzv. \emph{Afinita}. Afinita je index určující vhodnost reklamního nosiče, vyjádřená jako poměr sledovanosti média cílovým publikem
a obecné populace. Čím vyšší afinita, tím vyšší možnost oslovení cílového publika.

\subsection{Soudržná komunikace}
Posledníl, neméně důležitá podmínka pro úšpěšnou kampaň, je soudržná prezentace brandu
\footnote{Brand označuje značku organice, výrobku nebo služby. Pod brandem je myšleno jak grafické vyjádření, tak i podstata toho, co znázorňuje.}
nebo promovaného zboží. Dobrá komunikace vede k lepšímu rozpoznání potenciálním zákazníkem a zdůrazňuje, že zde je zákazník na správném místě.

\section{Internetové kampaně}
Různé studie neustále poukazují na čím dál více se zvyšující čas jednotlivcem strávený na Internetu. Pro spoustu inzerentů to znamená vyšší možnost
zaujmutí zákazníka a celkově větší možné publikum. Internet se tímto stal velice atraktivním médiem a to nejen díky vysoké míře \emph{OTS}
\footnote{Opportunity to see, průměrná možnost zhlédnutí reklamy příslušníkem cílového publika během kampaně.},
ale také jednoduché možnosti cílení a monitorování úspěšnosti reklamy.


\subsection{Poskytované typy sítí}
Pro spoustu uživatelů Internetu se jejich pomyslnou bránou do tohoto digitálního světa staly vyhledávače. Není divu, že se tyto vyhledávače staly jedním z největších
poskytovalů online reklamy. Z toho plyne i první typ reklamní sítě. A tou je právě \emph{Vyhledávácí síť}. Reklama zde bývá zobrazena ve formě textových odkazů,
vedoucí na webové stránky inzerenta. Ač se jedná o jednoduchou a negrafickou reklamu, jedná se o jednu z nejúspěšnějích forem reklamy, co se prokliků týče. Tato skutečnost
je způsobena tím, že se uživateli zobrazují výsledky spojené s tím, co si zahledal, tedy pro něj mají vysokou relevanci.

Dalším typem jsou \emph{Obsahové sítě}. Jedná se většinou o sdružení partnerských webů, které se snaží o jednotnou identifikaci uživatele. Reklama na síti tohoto typu bývá
grafická, ve formě bannerů, videoreklamy, nebo celostránkové reklamy\footnote{Sklik tyto možnost nabízí pod názvem \enquote{Branding}}.

Jako třetí typ se autor rozhodl uvést také \emph{Sociální sítě}. Ty na základě zájmů svých uživatelů, sociálních grafů a podobných technik nabízejí dobře cílenou inzerci.
Ta může být grafická a zároveň interaktivní. Nejčastějším trendem je reklama vložená do tzv. \enquote{stories}.

\subsection{Typy kampaní}
Typy kampaní jsou závislé na systému, které je poskytuje. Typicky ale lze očekávat podporu například videokampaní, obsahových kampaní nebo taky produktové kampaně.
Typy kampaně ovlivňuje, kde se bude reklama zobrazovat a v jaké formáty podporuje. 

\subsection{Reklamní formáty}
Formáty můžeme rozdělit na textové a grafické. Zatímco u textových formátů toho nelze moc vylepšit, grafické formáty se zobrazují v různých variantách:
\begin{itemize}
    \item Videospot -- přeskočitelná video reklama.
    \item Bumper -- nepreskočitelná krátká video reklama.
    \item Banner -- obrázek s nabídkou.
    \item Branding -- celostránková reklama.
    \item HTML5 banner -- reklama vytvořená pomocí webových technologií.
    \item Karusel -- posuvné reklamy s více obrázky.
\end{itemize} 

\subsection{XML feed}

\chapter{Analýza systémů pro správu kampaní}
Jelikož jednoduchá správa reklamních kampaní je problém existující delší dobu, byly již vytvořeny nástroje ulehčující tuto úlohu.
V této kapitole je popsáno několik z těchto možných řešení se zaměřením na usnadnění práce při správě kampaní.


\section{Sklik}
Český Sklik provozuje svůj systém ve formě webové aplikace. Nabízí import a export kampaní 
(ve formátu CSV\footnote{Ačkoliv formát naznačuje oddělení polí čárkami, ve skutečnosti jedná spíše o znak tábulátoru.})
pro zálohování nebo přenos mezi dalšími systémy. Takto exportovaný soubor má výhodu přímé a rychlé editace dat, ale je podmíněn nutnou znalostí významu jednotlivých
sloupců. Při importu je možné využít nastavení pro přepočet měny, najít a nahradit, filtr na určité sloupce a typ import, tedy zda má import kampaně přepsat, aktualizovat
nebo duplikovat.
Další funkcí jsou návrhy klíčových slov. Tato funkce uživateli zobrazí hledanost klíčového slova (počet zahledání za období), roční trend, velikost konkurence a průměrnou
cenu za proklik reklamy zobrazenou na dané klíčové slovo. Poslední a velice užitečná funkce jsou automatická pravidla. Automatická pravidla se mohou nastavit buďto na
celé sestavy nebo jednotlivá klíčová slova. Jejich účel je automatické zvyšování/snižování ceny za proklik na zákládě například: počtu prokliků, zobrazení konverze, atd.
Tímto dynamickým ovládáním ceny za reklamu si inzerenti mohou v případě dobré výkonnosti sestavy/klíčového posílit svou pozici oproti konkurenci, nebo snížit náklady
v případě, kdy se nedaří.



\section{Google Ads Editor}
Nástroj vytvořený společností Google je desktopová aplikace s možností správy nejen kampaní, ale také inzerentních účtů na platformě Google Ads. Využitím operace
\emph{drag and drop} lze jednoduše kopírovat sestavy a kampaně a další nastavení. Při vytváření kampaní může uživatel zadat nebo vybrat nesprávné kombinace určitých
nastavení, je však na ně ihned upozorněn a nucen je opravit. Pro nastavení ceny nabídek se namísto automatických pravidel,
uplatňuje tzv. \emph{strategie nabídek} (angl. Bid strategy). Zde uživatel může vybrat pevně danou cenu za proklik, nebo strategii na maximalizaci prokliků, konverzí
a další. Těmto strategiím se dají nastavit stropní hodnoty. Dalším zajimavým nastavením je také rotování reklam, tedy jak často se budou zobrazovat reklamy v této
kampani, podle toho, jak se jim daří. Většina ostatních nastavení lze rychle vyplnit a vytvořit sestavy a reklamy.

\section{Mergado}
Produktové kampaně využivají primárně data z XML feedu. Často se však stává, že je potřeba ve feedu udělat udělat nějaké úpravy, validovat správnou strukturu nebo feed
optimalizovat. Platforma Mergado umožňuje provádět tyto změny v XML feedu z prostředí prohlížeče bez jakéhokoli zásahu programátora. Tento upravený feed je poté
poslán na srovnávače zboží (Heureka, Zboží.cz, \ldots), případně na inzerentní systémy jako například Sklik. Mergado přes svůj \emph{App Cloud} umožňuje například
upravení cen produktů v XML v feedu v závislosti na ceně konkurentů, automatickou úpravu biddingu nebo dokonce obrázkovou úpravu produktovýchg náhledů.

\section{XeMeL}
Další nástroj pro úpravu XML feedů. Mezi jeho silné stránky spadá to, že XML feedy aktualizuje každých 45 minut, dokáže spojit více feedů do jednoho, případně
se přizpůsobí i na odlišené vstupní feedy. 

\chapter{Návrh}

\section{Tvorba kampaní}
Možnost tvorby kampaní se může rozložit buďto na jednolivé systémy nebo vytvořením "Super kampaně." Super kampaň bude mít možnost rozložení mezi více platforem pro správu kampaní.
Pro každý z těchto systémů budou různé možnosti nastavení parametrů kampaně, podle podpory funkcí na daném systému. Příkladem může být nastavení denní rozpočtu, geologické cílení atd.
Výhodou takové nastavení bude maximální kontrola nad kampaněmi, ale všechny potřebné věci (bannery, klíčová slova a další) budou spravovány na jednom jednoduchém rozhraní.

\chapter{Implementace}


% Seznam literatury
\printbibliography[title={Literatura}, heading=bibintoc]
% Zde vlozime uvod

\end{document}