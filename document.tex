\documentclass[czech,bachelor]{diploma}
% Dalsi doplnujici baliky maker
\usepackage[autostyle=true,czech=quotes]{csquotes} % korektni sazba uvozovek, podpora pro balik biblatex
\usepackage[backend=biber, style=iso-numeric, alldates=iso]{biblatex} % bibliografie
\usepackage{dcolumn} % sloupce tabulky s ciselnymi hodnotami
\usepackage{subfig} % makra pro "podobrazky" a "podtabulky"
\usepackage[cpp]{diplomalst}

% Zadame pozadovane vstupy pro generovani titulnich stran.
\ThesisAuthor{Bc. Lukáš Moravec}

\ThesisSupervisor{Ing. Radoslav Fasuga, Ph.D.}

\CzechThesisTitle{Systém pro tvorbu podkladu pro on-line reklamní kampaně}

% \EnglishThesisTitle{Tools for Automated Generation of Promo Graphic Sources and Videos}

\SubmissionYear{2022}


% \Acknowledgement{Rád bych na tomto místě poděkoval vedoucímu bakalářské práce, Ing.~Radoslavovi~Fasugovi,~Ph.D., za věnovaný čas, vstřícnost a ochotu ke konzultacím při vytváření této práce. Dále bych chtěl také poděkovat své přítelkyni a rodině za neustálou podporu.}

% \CzechAbstract{Tato bakalářská práce je cílená na problematiku automatizovaného generování reklamních bannerů. V první části se zaměřuje na to, co to vlastně reklama je, jak funguje a nejčastější způsoby toho, jak využít různé formy reklamy k dosažení svých cílů. Dále popisuje tvorbu bannerů a podmínky pro nasazení do různých reklamních sítí. Práce analyzuje a porovnává existující nástroje k tvorbě reklamních materiálů, jejich možnosti automatizace tvorby, následně pojednává o tom, jak vytvořit nástroj, který by automatizaci tvorby bannerů umožňoval a popisuje důležité funkce úprav nezbytné k efektivnímu vytváření reklamních materiálů. Výsledná aplikace je jednostránková webová aplikace, sestavená za pomoci aplikačního rámce Angular a knihovny Konva.js.}

% \CzechKeywords{Reklama; Banner; Online grafika; Angular; Konva.js; SPA}

% \EnglishAbstract{This bachelor's thesis is aimed at the problematics of automated generation of advertising banners. In first part, the thesis focuses on what an advertisement really is, how it works and the most common ways of how to use different forms of advertising to achieve it's goals. It also describes the creation of banners and conditions for deployment in advertising networks. It analyses and compares existing tools to create advertising materials and their options of automated creation. It then discusses on how to create a tool, which would enable automated creation of banners. The thesis describes important editing features necessary to effective creation of advertising materials. The final application is a single page web application build with Angular framework and Konva.js library}

% \EnglishKeywords{Advertisement; Banner; Online graphics; Angular; Konva.js; SPA}

\addbibresource{bibliography.bib}

% Novy druh tabulkoveho sloupce, ve kterem jsou cisla zarovnana podle desetinne carky
\newcolumntype{d}[1]{D{,}{,}{#1}}

% Zacatek dokumentu
\begin{document}

% Nechame vysazet titulni strany.
\MakeTitlePages

\chapter{Úvod}
Možnosti internetové inzerce převyšujicí tradiční formy reklamy. Vystavit svou firmu pozornosti až tisíců potenciálních zákazníků během pár hodin je vysoké lákadlo pro investující firmy. Neustále zvyšující se nároky inzerentů stojí za vznikem komplexních systémů
pro správu online reklamních kampaní, které umožnují mít své nastavení pod naprostou kontrolou svých uživatelů. Aby dosah reklamy byl co nejširší, vyplatí se inzerovat na více platformách najednou. Těmi největšími platformami se staly 
\emph{Google Ads, Facebook} nebo český \emph{Sklik}. Všechny tyto systémy poskytují jak své vlastní řešení pro správu kampaní, tak i aplikační rozhraní pro možnosti automatizace častých úkonů.
Aplikačním rozhraním se ale také otevírá brána pro vykonávání komplexních úkolů včetně kompletní správy kampaní napříč systémy. Tedy z jednoho místa, ovládat inzerci na několika platformách najednou.
Tato semestrální práce se zabývá vytvořením takovéhoto nástroje v online prostředí, který by umožňoval jednoduchou správu reklamních kampaní na více inzerčních systémů. 


\chapter{Reklamní kampaně}
Reklamní kampaň lze definovat jako soubor komunikačních a reklamních aktivit, které mají za úkol oslovit veřejnost a splnit zvolený cíl. K uskutečnění kampaně se dá využít různých komunikačních
médií (internet, rozhlas, televize, \ldots). Její úspěšnost zavisí také na jejím správném provedení. Tato kapitola popisuje několik stěžejních bodů, které tvoří základ při tvorbě reklamní kampaně.


\section{Cíl kampaně}
Zvolení cíle zavisí na tom, co chceme kampaní dosáhnout. Typicky se jedná o zvýšení prodejnosti zboží nebo služeb, to ale nejsou jediné možnosti. Další mohou být například budování
povědomí o své značce (brand) nebo předání nějaké informace. Znát cíl pomáhá s další tvorbou kampaně a udává její směr.

\section{Rozpočet}
Finanční možnosti jsou jedním ze základních kritérií, ne-li nejdůležitějším. Od rozpočtu se odvíjí nejen celý zbytek plánování reklamní kampaně, ale i možná 
doba trvání kampaně. Je velice pravděpodobné, že malý rozpočet se rychleji vyčerpe a proto je nutné kampaň vhodně rozložit.

\section{Publikum}
Cílové publikum je další důležitý faktor. Tato volba může znamenat rozdíl mezi dobrou nebo špatnou investicí. Různá sociodemografická kritéria pomohou lépe porozumět
na koho se kampaň směřuje. Toho lze využít při vytváření plánů komunikace, reklamy, nebo volby média.

\section{Média}
Různá média, různě působí na zvolené cílové publikum. I proto je dobře specifikované cílové publikum důležité. Pokud je cilovým publikem například starší populace, investice
do PPC reklamy může být značně nevýhodná. Dalším kritériem pro výběr komunikačního kanálů mohou být možnosti geologického cílení, zpětné vazby a statisky úspěšnosti. 


\section{Soudržná komunikace}


\chapter{Analýza existujících nástrojů}
Tato kapitola se zabývá, jak název naznačuje, analýzou existujících systémů pro správu reklamních kampaní, rozebírá jejich možnosti a rozdíly. 


\section{Sklik}

\section{Google Ad Manager}

\section{Facebook}


\chapter{Návrh}

\section{Tvorba kampaní}
Možnost tvorby kampaní se může rozložit buďto na jednolivé systémy nebo vytvořením "Super kampaně." Super kampaň bude mít možnost rozložení mezi více platforem pro správu kampaní.
Pro každý z těchto systémů budou různé možnosti nastavení parametrů kampaně, podle podpory funkcí na daném systému. Příkladem může být nastavení denní rozpočtu, geologické cílení atd.
Výhodou takové nastavení bude maximální kontrola nad kampaněmi, ale všechny potřebné věci (bannery, klíčová slova a další) budou spravovány na jednom jednoduchém rozhraní.  

\chapter{Implementace}


% Seznam literatury
\printbibliography[title={Literatura}, heading=bibintoc]
% Zde vlozime uvod

\end{document}